%! Author = mariuszindel
%! Date = 02.11.20

\section{Vererbung}

\subsection{Überblick}
\begin{itemize}
    \item Nur eine Basisklasse erlaubt
    \item Beliebig viele Interfaces erlaubt
    \item Structs können nicht erweitert werden
    \item Structs können nicht erben
    \item Structs können nur Interfaces implementieren
    \item Klassen sind direkt / indirekt von System.Object abgeleitet
    \item Structs sind über Boxing mit System.Object kompatibel
\end{itemize}
\subsubsection{Zuweisungen}
Zuweisung immer möglich wenn
\begin{itemize}
    \item Statischer Typ gleich Dynamischer Typ
    \item Statischer Typ Basisklasse von Dynamischem Typen
\end{itemize}
Zuweisung verboten wenn
\begin{itemize}
    \item Statischer Typ Subklasse von Dynamischem Typen
    \item Statischer Typ und Dynamischer Typ nicht in der gleichen Vererbungshierarchie sind
\end{itemize}
\begin{lstlisting}
class Base { }
class Sub : Base { }
class SubSub : Sub { }

public static void Test()
{
// Statischer Typ: Base
// Dynamischer Typ: Base
    Base b = new Base();

// Dynamischer Typ: Sub
    b = new Sub();

// Dynamischer Typ: SubSub
    b = new SubSub();

// Compilerfehler (Typecast)
    Sub s = new Base();
    }
\end{lstlisting}
\subsubsection{Typprüfungen}
Prüft ob ein Objekt mit einem Typen kompatibel ist $\rightarrow$ Liefert bool\\

Liefert true wenn:
\begin{itemize}
    \item Typ von «obj» identisch wie «T» ist
    \item Typ von «obj» eine Sub-Klasse von «T» ist
\end{itemize}
Liefert false wenn
\begin{itemize}
    \item «obj» null ist
    \item Typ von «obj» eine Basisklasse von «T» ist
    \item Typ von «obj» und «T» nicht in der gleichen Vererbungshierarchie sind
\end{itemize}
\begin{lstlisting}
if (a is SubSub) { /* ... */ }
\end{lstlisting}

\subsubsection{Type Casts}
Explizite Typumwandlung als Hinweis für den Compiler
\begin{itemize}
    \item Anweisung an Compiler, einen Typen in einen anderen umzuwandeln
    \item null kann auch gecastet werden
    \item Compilerfehler wenn erkennbar, dass Type Cast nicht zulässig
    \item Compilerfehler wenn erkennbar, dass null in einen Value Type gecastet wird
    \item Laufzeit-Fehlverhalten: InvalidCastException
\end{itemize}
\begin{lstlisting}
Sub s = (Sub)b;
\end{lstlisting}

\subsubsection{Type Casts mit as-Operator}
ExpliziteTypumwandlungals Hinweis für den Compiler
\begin{itemize}
    \item Syntactic Sugar / Compiler erzeugt «obj is T ? (T)obj : (T)null»
    \item Laufzeit-Fehlverhalten: null
\end{itemize}
\begin{lstlisting}
Sub s = b as Sub;
\end{lstlisting}

\subsubsection{Typprüfungen mit implizitem Type Cast}
PrüftobeinObjektmiteinemTypen kompatibel ist
\begin{itemize}
    \item Syntax: «obj is T result»
    \item Variable «result» vom Typ «T» enthält gecasteten Wert
    \item Funktioniert wie normaler Type Cast
    \item Arbeitet nicht mit Nullable Types im Gegensatz zu «as»
\end{itemize}
\begin{lstlisting}
Base a = new SubSub();
if (a is SubSub casted) {
Console.WriteLine(casted); }
\end{lstlisting}


\subsection{Methoden \& Vererbung}
\subsubsection{Methoden überschreiben}
Subklasse kann Members der Basisklasse überschreiben.\\
$\rightarrow$ Schlüsselwörter «virtual» und «override»

\begin{itemize}
    \item Basisklasse:\\
    «virtual» um Basis-Methode überschreibbar zu machen
    \item Subklasse:\\
    «override» um Basis-Methode zu überschreiben
\end{itemize}
\begin{lstlisting}
class Base {
public virtual void G() { /* ... */ } }
class Sub : Base
{
public override void G() { /* ... */ } }
\end{lstlisting}
Regeln:
\begin{itemize}
    \item Members sind per Default NICHT «virtual»
    \item Members sind per NICHT «overrride»
    \item Signatur muss identisch sein
\end{itemize}

\subsubsection{Methoden überdecken}
\begin{itemize}
    \item Unterbricht Dynamic Binding!
    \item Subklasse kann Members der Basisklasse überdecken
\end{itemize}
\subsubsection{Methoden überdecken mit «new»}
\begin{itemize}
    \item Compiler warnung kann mit «new» Schlüsselwort vermieden werden
    \item Hinweis für Compiler, dass Member bewusst überdeckt wurde
\end{itemize}
\begin{lstlisting}
class Base {
public virtual void I() { /* ... */ } }
class Sub : Base
{
public override void I() { /* ... */ } }
class SubSub : Sub
{
public new void I() { /* ... */ }
public new void I2() { /* ... */ } }
\end{lstlisting}

\subsubsection{Dynamic Binding / Regelwerk komplett in Pseudocode}
Suche Vererbungs-Hierarchie von oben nach unten nach konkretester Methode «G()» mit Schlüsselwort «override»


\subsection{Abstrakte \& Versiegelte Klassen}
\subsubsection{Abstrakte Klassen}
\begin{itemize}
    \item Mischung aus Klasse und Interface
    \item Kann nicht direkt instanziert werden
    \item Kann beliebig viele Interfaces implementieren
\end{itemize}
\begin{lstlisting}
abstract class Sequence {
public abstract void Add(object x); // Method
public abstract string Name { get; } // Prop.
public abstract object this[int i] // Idx.
{ get; set; }
public abstract event EventHandler OnAdd;
                                         // Event

public override string ToString() { return Name; }
}
class List : Sequence
{
public override void Add(object x) { /* ... */ }
public override string Name
{ get { /* ... */ } }
public override object this[int i]
{ get { /* ... */ } set { /* ... */ } } public override event EventHandler OnAdd;}
\end{lstlisting}
\subsubsection{Abstrakte Methoden}
\begin{itemize}
    \item Kein Anweisungsteil
    \item Sind implizit virtual
    \item Dürfen nicht «static» oder «virtual» sein
\end{itemize}
\subsubsection{Abstrakte Properties / Indexer}
\begin{itemize}
    \item Kein Anweisungsteil
    \item Sind implizit virtual
    \item Dürfen nicht «static» oder «virtual» sein
    \item «get» und «set» Kombination muss bei Implementation identisch sein
\end{itemize}

\subsubsection{Versiegelte Klassen}
\begin{itemize}
    \item Verhindert das Ableiten einer Klasse
    \item Analog «final» bei Java
    \item Sicherheitsaspekte verhindert versehentliches Erweitern
\end{itemize}

\begin{lstlisting}
sealed class Sequence {
    public void Add(object x) {}
    public string Name { get{} }
    public object this[int i] { get{} }
    public event EventHandler OnAdd;
}
// Compilerfehler
class List : Sequence { /* ... */ }
\end{lstlisting}


\subsubsection{Versiegelte Members}
\begin{itemize}
    \item Verhindert das Überschreiben eines bestimmten Members
    \item Muss in Kombination mit «override» angewendet werden
    \item Kombination mit «virtual» und «new» ist nicht erlaubt
\end{itemize}
\begin{lstlisting}
class List {
    public sealed void Add(object x) {}
}
class MyList : List
{ // Compilerfehler
    public override void Add(object x) {}
}
\end{lstlisting}

\subsection{Interfaces}
\begin{itemize}
    \item Interface ähnelt einer rein abstrakten Klasse
    \item Kann nicht direkt instanziiert werden
    \item Interface kann andere Interfaces erweitern
    \item Sichtbarkeit auf Members darf nicht angegeben werden
    \item Members sind implizit «abstract virtual»
    \item Members dürfen nicht «static» sein oder ausprogrammiert werden
    \item «override» ist nicht nötig
\end{itemize}
\begin{lstlisting}
interface ISequence {
void Add(object x); // Method
string Name { get; } // Propert
object this[int i] { get; set; } // Indexer
event EventHandler OnAdd; // Event
\end{lstlisting}


\subsection{Interfaces / NamingClashes}
\begin{itemize}
    \item Zwei Interfaces können gleiche Members definieren
    \item Führt zu Kollisionen
    \item Lösung: Explizites implementieren des Interfaces
\end{itemize}

Lösungsansatz mit Default Variante:
\begin{lstlisting}
class ShoppingCart : ISequence, IShoppingCart
{
    public void Add(object x) { /* ... */ }

    void IShoppingCart.Add(object x) { /* ... */ }
}

// Anwendung:
ShoppingCart sc = new ShoppingCart(); sc.Add("Hello"); // Add
ISequence sc1 = new ShoppingCart(); sc1.Add("Hello"); // Add
IShoppingCart sc2 = new ShoppingCart(); sc2.Add("Hello"); // IShoppingCart.Add
\end{lstlisting}

\newpage