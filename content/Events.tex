%! Author = mariuszindel
%! Date = 02.11.20

\section{Events}

\subsection{Übersicht}
\begin{itemize}
    \item Reines Compiler-Feature / Syntactic Sugar
\end{itemize}

\subsection{Callback-Tick}
\begin{itemize}
    \item «OnTickEvent» Delegate wird private
    \begin{itemize}
        \item Kann also nur noch intern ausgelöst werden
    \end{itemize}
    \item Subscribe / Unsubscribe Logik wird public generiert
    \item Observer bleibt identisch
\end{itemize}

\begin{lstlisting}
public delegate void TickEventHandler (int ticks, int interval);

public class Clock { £(Notifier/observer)£
    public event TickEventHandler OnTickEvent;

    private void Tick(object sender, EventArgs e) {
    ticks++; OnTickEvent?.Invoke(ticks, interval); }
}
\end{lstlisting}


\subsection{Event Handling im .NET Framework}

Standard-Syntax bei Events:
\begin{lstlisting}
public delegate void AnyHandler(object sender, AnyEventArgs e);
\end{lstlisting}

\begin{itemize}
    \item Rückgabetyp void
    \item 1. Parameter object sender
    \begin{itemize}
        \item Sender des Events
        \item Absender übergibt bei Aufruf des Delegates / Events «this» mit
    \end{itemize}
    \item 2. Parameter AnyEventArgs e
    \begin{itemize}
        \item Beliebige Sub-Klasse von «EventArgs»
        \item Enthält Informationen zum Event
        \item Argumente können jeder ergänzt werden ohne Anpassung der Signatur
    \end{itemize}
\end{itemize}

\subsection{Anonyme Methoden}

\subsubsection{Übersicht}
\begin{itemize}
    \item Methodencode wird in-place angegeben
    \item Keine Deklaration einer Methode nötig
    \item Anonyme Methode kann auf lokale Variable «sum» zugreifen
    \item Return beendet anonyme Methode, nicht die äussere Methode
\end{itemize}
\begin{lstlisting}
class AnonymousMethods {
    void Foo() {
        list.ForEach(delegate(int i)
            { Console.WriteLine(i); }
);

    int sum = 0;
    list.ForEach(delegate(int i)
    { sum += i; });
    }
}
\end{lstlisting}

\subsubsection{Closures / Outer Variables (Äussere Variablen)}
\begin{itemize}
    \item Anonyme Methode greift auf Variable der umgebenden Methode zu
    \item Variable wird in Hilfsobjekt ausgelagert
    \item Hilfsobjekt gleich lange wie Delegate-Objekt
    \item Adder ist eine Closure
\end{itemize}